\section{Challenges}
\label{sec:challenges}


\subsection{Lab computer setup (CG)}
% - /var/tmp not mounted
% - TFTP boot not working
% - Missing libs for linux build
% - No administration rights on lab pc
% 	- Can't be solved on our own
Most of our work was done on the lab computer. During this work we encountered several challenges which proved at least time consume. The lab computer is attached to the lrz network and on booting it loads an image from a central server. Because of this network set up the user has very restricted rights on the computer to prevent abuse. However this also prevented us from installing missing libraries on the computer e.g. mosquitto for gaining experience and testing of mqtt communication. 

\subsection{Build infrastructure issues (FM)}
% - Project setup with updated provision scripts


\subsection{Raspberry Pi build (TM)}
\label{sec:pi-problems}
Until the last two weeks of the lab course we were not able to build the Genode operating system for the Raspberry Pi. According to the documentation of the \textit{argos-research} project\footnote{\url{https://argos-research.github.io/platforms.html}} the only requirement in order to build the image for the Raspberry Pi is to change the \texttt{GENODE\_TARGET} inside the Makefile to \texttt{foc\_rpi}. However, due to problems in the kernel of Genode itself, the compilation always failed. \\

Since our knowledge about the internals of Genode is not that high, we were not able to fix this problem by ourselves. However, trying to compile and searching for the problem still took quite a long time of our lab course. \\

Apparently, after some time the problem was fixed by a member of the \textit{argos-research} team by updating the Genode version and changing from the \texttt{foc} kernel to the \texttt{focnados}. However, this way we were able to compile the image, but the image was not bootable on the hardware. Later, it turned out that a wrong base address inside the bootlader was causing this problem. Again we were not able to solve this issue by ourselves. \\

In the meantime we decided to switch to \textit{Raspbian} instead of Genode since we could not predict whether we would be able to compile the operating system until the end of the lab course, at all. \\

In the last two weeks of the course the problem was again fixed by the member of the \textit{argos-research} team and we could finally switch back to Genode.
