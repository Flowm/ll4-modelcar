\section{Challenges}
\label{sec:challenges}


\subsection{Lab computer setup (CG)}
% - /var/tmp not mounted
% - TFTP boot not working
% - Missing libs for linux build
% - No administration rights on lab pc
% 	- Can't be solved on our own
Most of our work was done on the lab computer. During this work we encountered several challenges which proved to be at least time consuming. The lab computer is attached to the lrz network and on booting it loads an image from a central server. Because of this network set up the user has very restricted rights on the computer to prevent abuse.\\

However this also prevented us from installing missing libraries on the computer, e.g. mosquitto for gaining experience and testing of mqtt communication. Furthermore our working directory(/var/tmp) was not mounted at some point for an unknown reason. Because the working directory was on the local harddrive, everything had to be set up again on the second lab computer and the latest changes were lost.\\

Additionally it was planned to use tftp boot to ease booting a newly compiled image. However, this first failed due to firewall rules and later due to missing administrator rights on the computer.\\

All these problems could not be fixed by ourselves, but needed either a workaround or a technician. Nevertheless trying to fix them ourselves cost us a lot of time. For example getting tftp boot to work took us a whole afternoon of three hours although we were getting help from the system administrator of the chair. Moreover it would have saved us time later, especially as the sd card reader was not reliably working.


\subsection{Raspberry Pi build (TM)}
\label{sec:pi-problems}
Until the last two weeks of the lab course we were not able to build the Genode operating system for the Raspberry Pi. According to the documentation of the \textit{argos-research} project\footnote{\url{https://argos-research.github.io/platforms.html}}, the only requirement in order to build the image for the Raspberry Pi is to change the \texttt{GENODE\_TARGET} inside the Makefile to \texttt{foc\_rpi}. However, due to problems in the kernel of Genode itself, the compilation always failed. \\

Since our knowledge about the internals of Genode is limited, we were not able to fix this problem by ourselves. However, trying to compile and searching for the problem still took quite a long time of our lab course. \\

Apparently, after some time the problem was fixed by a member of the \textit{argos-research} team by updating the Genode version and changing from the \texttt{foc} kernel to the \texttt{focnados}. However, this way we were able to compile the image, but the image was not bootable on the hardware. Later, it turned out that a wrong base address inside the bootloader was causing this problem. Again we were not able to solve this issue by ourselves. \\

In the meantime we decided to switch to \textit{Raspbian} instead of Genode since we could not predict whether we would be able to compile the operating system until the end of the lab course, at all. \\

In the last two weeks of the course the problem was again fixed by the member of the \textit{argos-research} team and we could finally switch back to Genode.


\subsection{Build infrastructure (FM)}
% - Project setup with updated provision scripts
The \textit{argos-research/operating-system} repository provides a Varantfile in combination with a provision script allowing the setup of the build environment in an automated and reproducible manner.
Unfortunately the vagrant setup caused some issues with multiple project members due to the use of Virtualbox shared folders for the complete source code repository.
Additionally the provision script could not be used in the lab environment due to missing access right for some steps that required elevated permissions and some hardcoded paths only available in vagrant. \\

After an update of the provision script based on feedback during the first weeks of the practical it can now also be used to setup up the project outside of the lab environment and without direct dependence on vagrant as virtualization backend.

To ensure a simple setup of the model-car project with all it's dependencies without direct integration in the \textit{argos-research/operating-system} repository a \texttt{setup.sh} script is provided in the top level of the project repository and the required steps for the build are described in the projects \texttt{README.md}.
