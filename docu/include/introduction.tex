\section{Introduction}
This document describes the work of team \textit{model\_car} for the LL4 lab course at \textit{Technische Universität München} in summer 2017. \\

The overall goal of the project is to set up a hardware-in-the-loop (HIL) scenario based on Genode with Fiasco.OC. An image of the scenario is shown in figure \ref{fig:hil}. As one can see, the software part of the HIL is a racing game called \textit{SpeedDreams} which sends control commands to a physical model of a car. The concrete commands sent to the model are braking-, steering-, and acceleration requests. The model car performs the requests in hardware and returns sensor values to the simulation. \\

\begin{figure}[h]
    \centering
    \includegraphics[width=0.7\linewidth]{images/hil}
    \caption{HIL scenario with SpeedDreams and a physical model of a car}
    \label{fig:hil}
\end{figure}

This document describes the hardware part of the HIL, whereas the software part is implemented by another team of the lab course.
