\subsection{PandaBoard (FM)}
\label{sec:panda}
%TODO FM: Write chapter

\subsubsection{Overview}
\label{sec:panda-overview}



\subsubsection{Requirements}
\label{sec:panda-req}
The following tasks were derived from the initial project description for the component running on the PandaBoard:
\begin{itemize}
    \item \textbf{T1}: Install Genode with Fiasco.OC
    \item \textbf{T2}: Develop MQTT client
    \item \textbf{T3}: Convert control commands into concrete servo values
    \item \textbf{T4}: Generate control commands
\end{itemize}

\subsubsection{Hardware}
\label{sec:panda-hw}

\subsubsection{Genode application}
\label{sec:panda-genode}

\subsubsection{Command processing}
\label{sec:panda-convert}

\subsubsection{Testing}
\label{sec:panda-testing}
To verify the data processing and the actuation of the hardware servos based on control commands sent to the \texttt{car-control} MQTT topic a test script was developed.
This way the hardware side of the hardware in the loop setup described in \autoref{sec:intro} can be tested independently of the software side developed by another project team.

The publish–subscribe pattern used by the MQTT server allows multiple clients to simultaneously connect to the server and subscribe to updates on a certain topic or publish messages to a certain topic.
In this style of communication there are no direct communication links and the components are loosely coupled, simplifying the testing of the interfaces between components.

By using the MQTT commandline tools available under Linux in the \texttt{mosquitto-clients} package message can easily be injected into the \texttt{car-control} topic without changing the configuration of any components.
\autoref{lst:panda-msgpub} shows a small test script that continiously sends steering control messages that cause the steering servos to alternate between actuation to the left and to the right. \\ % when processed by Panda, Rpi and servo components.

\begin{minipage}{\linewidth}
\begin{lstlisting}[style=mylistings, language=c, label=lst:panda-msgpub, caption=Injecting steering commands over MQTT]
#!/usr/bin/env bash
MQTT_IP=$1

while true; do
	mosquitto_pub -h $MQTT_IP -t 'car-control' -m '0,1.0'
	sleep 1
	mosquitto_pub -h $MQTT_IP -t 'car-control' -m '0,-1.0'
	sleep 1
done
\end{lstlisting}
\end{minipage} \\

In a similar manner the \texttt{mosquitto\_sub} commandline tool can be used to subscribe to MQTT topics. This provides easy access to the communication between components for debugging and testing purposes.
\autoref{lst:panda-msgsub} shows the command used to monitor the message published by the panda component. In both listings \texttt{\$MQTT\_IP} is a variable containing the IP address of the MQTT server. \\

\begin{minipage}{\linewidth}
\begin{lstlisting}[style=mylistings, language=c, label=lst:panda-msgsub, caption=Subscribing to MQTT topics]
mosquitto_sub -h $MQTT_IP -t 'car-servo'
\end{lstlisting}
\end{minipage} \\
